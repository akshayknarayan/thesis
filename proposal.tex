%% This file is for producing a Doctoral Thesis proposal.  It should be fairly
%% self-explanatory.  

\documentclass{article}

\usepackage{xspace}
\usepackage{cite,url}

\begin{document}

%\bibliographystyle{plain}
\pagestyle{empty}
\markboth{{\sc thesis proposal}}{{\sc thesis proposal}}
\def\title{Design and Implementation of a Congestion Control Plane}
\def\author{Akshay Narayan}
\def\addrone{32 Vassar St 32-G982}
\def\addrtwo{Cambridge, MA  02139}

\def\degree{Master of Science}
%% Added \deptname for PhD proposals in other fields.
%% Krishna Sethuraman (1990)
\def\deptname{Electrical Engineering \\ and Computer Science}
\def\laboratory{Computer Science and Artificial Intelligence Laboratory}

\def\submissiondate{\today}
\def\completiondate{February 2018}

\def\supervisor{Professor Hari Balakrishnan}
\def\supertitleone{Associate Professor of Electrical Engineering}
\def\supertitletwo{and Computer Science}

\def\readerone{Professor Mohammad Alizadeh}
\def\readeronetitleone{Professor of Electrical Engineering}
\def\readeronetitletwo{and Computer Science}

\def\readertwo{}
\def\readertwotitleone{}
\def\readertwotitletwo{}

\def\readerthree{}
\def\readerthreetitleone{}
\def\readerthreetitletwo{}


\def\abstract{
With Moore's law ending, the gap between general-purpose processor
speeds and network link rates is widening. This trend has led to new
packet-processing ``datapaths'' in endpoints, including kernel
bypass software and emerging SmartNIC hardware.
In addition, several applications are rolling out their own protocols atop UDP (e.g.,
QUIC, WebRTC, Mosh, etc.), forming new datapaths different from the
traditional kernel TCP stack.
All these datapaths require congestion control,
but they must implement it separately because it is not possible
to reuse the kernel's TCP implementations. This paper proposes
moving congestion control from the datapath into a separate agent.
This agent, which we call the {\em congestion control plane\/} (CCP),
must provide both an expressive congestion control API as well as
a specification for datapath designers to implement and deploy CCP.\@
We propose an API for congestion control, datapath primitives, and a
user-space agent design that uses a batching method to communicate
with the datapath.
Our approach promises to preserve the behavior and performance of in-datapath
implementations while making it significantly easier to implement and deploy new
congestion control algorithms.
}

\newcommand{\cut}[1]{}
%
\newcommand{\paragraphnone}[1]{\vspace{0.075in}\noindent{\bf #1}}
\newcommand{\paragrapha}[1]{\vspace{0.075in}\noindent{\bf #1.}}
\newcommand{\paragraphq}[1]{\vspace{0.075in}\noindent{\bf #1}}
%
\newcommand{\eg}{e.g., }
\newcommand{\etc}{{etc.}\xspace}
\newcommand{\ie}{i.e., }
\newcommand{\ccp}{CCP\xspace}
%
\newcommand{\datapath}{datapath\xspace}
\newcommand{\datapaths}{datapaths\xspace}
\newcommand{\Datapaths}{Datapaths\xspace}
\newcommand{\Datapath}{Datapath\xspace}
%
\frenchspacing
%
\newcommand{\handlers}{callbacks\xspace}
\newcommand{\userspace}{user-space\xspace}
\newcommand{\kernelspace}{kernel-space\xspace}
%
\newcommand{\controller}{agent\xspace}

%%%%%%%%%%%%%%%%%%%%%%%%%%%%%%%%%%%%%%%%%%%%%%%%%%%%%%%%%%%%%%%%%%%%%%%%%%%%
%%%%%%%%%% You Should Not Need To Modify Anything Below Here %%%%%%%%%%%%%%%
%%%%%%%%%%%%%%%%%%%%%%%%%%%%%%%%%%%%%%%%%%%%%%%%%%%%%%%%%%%%%%%%%%%%%%%%%%%%

%%%%%%%%%%%%%%%%%%%%%%%%%%%%%%%%%
%%% Cover Page - Author signs %%%
%%%%%%%%%%%%%%%%%%%%%%%%%%%%%%%%%

\begin{center}
{\Large \bf 
   Massachusetts Institute of Technology
\\ Department of \deptname \\}
\vspace{.25in}
{\Large \bf
   Proposal for Thesis Research in Partial Fulfillment
\\ of the Requirements for the Degree of
\\ \degree \\}
\end{center}

\vspace{.5in}

\def\sig{{\small \sc (Signature of Author)}}

\begin{tabular}{rlc}
   {\small \sc Title:}                       & \multicolumn{2}{l}{\title}
\\ {\small \sc Submitted by:}
                            & \author  & \\
                            & \addrone & \\
                            & \addrtwo & \\ \cline{3-3}
			    &	       & \makebox[2in][c]{\sig}
\\ {\small \sc Date of Submission:}          & \multicolumn{2}{l}{\submissiondate}
\\ {\small \sc Expected Date of Completion:} & \multicolumn{2}{l}{\completiondate}
\\ {\small \sc Laboratory:}                  & \multicolumn{2}{l}{\laboratory}
\end{tabular}


\vspace{.75in}
{\bf \sc Brief Statement of the Problem:}

\abstract

         %%%%%%%%%%%%%%%%%%%%%%%%%%%%%
\newpage %%% Supervision Agreement %%%
         %%%%%%%%%%%%%%%%%%%%%%%%%%%%%

\begin{flushright}
   Massachusetts Institute of Technology
\\ Department of \deptname
\\ Cambridge, Massachusetts 02139
\end{flushright}

\underline{\bf Thesis Supervision Agreement}

\vspace{.25in}
\begin{tabular}{rl}
   {\small \sc To:}   & Department Graduate Committee
\\ {\small \sc From:} & \supervisor
\end{tabular}

\vspace{.25in}
The program outlined in the proposal:

\vspace{.25in}
\begin{tabular}{rl}
   {\small \sc Title:}  & \title
\\ {\small \sc Author:} & \author
\\ {\small \sc Date:}   & \submissiondate
\end{tabular}

\vspace{.25in}
is adequate for a \degree{} thesis.
%I believe that appropriate readers for this thesis would be:
%
%\vspace{.25in}
%\begin{tabular}{rl}
%   {\small \sc Reader 1:} & \readerone
%\\ {\small \sc Reader 2:} & \readertwo
%%\\ {\small \sc Reader 3:} & \readerthree
%\end{tabular}

\vspace{.25in}
Facilities and support for the research outlined in the proposal are available.
I am willing to supervise the thesis and evaluate the thesis report.

\vspace{.25in}
\begin{tabular}{crc}
  \hspace{2in} & {\sc Signed:} & \\ \cline{3-3}
               &               & {\small \sc \supertitleone} \\
               &               & {\small \sc \supertitletwo} \\
               &               &                             \\
               & {\sc Date:}   & \\ \cline{3-3}
\end{tabular}

\vspace{0in plus 1fill}

Comments: \\
\begin{tabular}{c}
  \hspace{6.25in} \\
  \mbox{} \\ \cline{1-1} \mbox{} \\
  \mbox{} \\ \cline{1-1} \mbox{} \\
  \mbox{} \\ \cline{1-1} \mbox{} \\
%  \mbox{} \\ \cline{1-1} \mbox{} \\
%  \mbox{} \\ \cline{1-1} \mbox{} \\
%  \mbox{} \\ \cline{1-1} \mbox{} \\
\end{tabular}

\newpage

\section{Background}%
\label{sec:intro}

Traditionally, applications have used the networking stack provided by the operating system through the socket API\@.  Recently, however, there has been a proliferation of new networking stacks as NICs have offered more custom features such as kernel bypass and as application developers have come to demand specialized features from the network~\cite{dpdk, mtcp, netmap, dcqcn, ix, quic}.  We refer to such modules that provide interfaces for data movement between applications and network hardware as \emph{datapaths}. 
Examples of datapaths include not only the kernel~\cite{lwn-pluggable-tcp}, but also kernel-bypass methods~\cite{dpdk, netmap, mtcp}, RDMA~\cite{dcqcn}, user-space libraries~\cite{quic}, and emerging programmable NICs (``SmartNICs''~\cite{smartnic}).

The proliferation of datapaths has created a significant challenge for deploying congestion control algorithms. For each new datapath, developers must re-implement their desired congestion control algorithms essentially from scratch---a difficult and tedious task. 
For example, implementing TCP on Intel's DPDK is a significant undertaking~\cite{mtcp}. We expect this situation to worsen with the emergence of new hardware accelerators and programmable NICs because many such high-speed datapaths forego programming convenience for performance.

Meanwhile, the diversity of congestion control algorithms---a key component of transport layers that continues to see innovation and evolution---makes supporting all algorithms across all datapaths even harder. The Linux kernel alone implements over a dozen~\cite{Jacobson88,newreno,cubic,vegas,illinois,htcp} algorithms, and many new proposals have emerged in only the last few years~\cite{dctcp,sprout,remy,pcc,rc3,gentleaggression,timely,bbr,dccp}. 
%It has become impractical to expect developers and researchers to implement a battery of congestion control algorithms for each new datapath. 
As hardware datapaths become more widely deployed, continuing down the current path will erode this rich ecosystem of congestion control algorithms because NIC hardware designers will tend to bake-in only a select few schemes into the datapath.

We believe it is therefore time for a new {\em datapath-agnostic} architecture
for endpoint congestion control. We propose to \emph{decouple} congestion
control from the datapath and relocate it to a separate agent that we call the  {\em congestion
  control plane} (\ccp{}). \ccp{} runs as a user-space program
  %\footnote{\ccp{} can also reside in the kernel but having it in user-space makes programming new algorithms much easier.} 
  and communicates asynchronously with the datapath to direct congestion control decisions. Our goal is to identify a narrow API for congestion control
that developers can use to specify flexible congestion control logic, and that
datapath developers can implement to support a variety of congestion control
schemes. This narrow API enables congestion control algorithms and
\datapaths to evolve independently, providing three key benefits:

\smallskip
\noindent
{\bf Write once, run everywhere.} With \ccp{}, congestion control researchers will be able to write a single piece of software and run it on multiple datapaths, including those yet to be invented. Additionally, datapath developers can focus on implementing a standard set of primitives---many of which are already used today---rather than worry about constantly evolving congestion control algorithms. 
Once an algorithm is implemented with the \ccp{} API, running on new \ccp{}-conformant datapaths will be automatic.

\smallskip
\noindent{\bf Ease of programming.} Locating \ccp in user-space and designing a convenient API for writing congestion control algorithms will make it easier to develop and deploy new schemes. 
This will ease the process of implementing and evaluating novel algorithms.
Furthermore, algorithm developers will be free to utilize powerful user-space libraries (\eg neural nets) and focus on the details of their methods rather than learning low-level datapath APIs.

% \smallskip
% \noindent{\bf Performance without compromises.} 
% As NIC line rates steadily march upwards to 100 Gbit/s and beyond, 
% there will be fewer CPU cycles available for complex congestion control
% operations. Removing congestion control from the \datapath will enable richer
% congestion control algorithms while retaining the performance of fast
% \datapaths. 
% \smallskip

\smallskip
\noindent{\bf Performance without compromises.} 
As NIC line rates steadily march upwards to 100 Gbit/s and beyond, it is
desirable to offload transport-layer packet processing to hardware to save CPU
cycles. Unlike solutions that hard-code a congestion control algorithm in
hardware, \eg ~\cite{dcqcn}, removing congestion control from the \datapath will
enable richer congestion control algorithms while retaining the performance of
fast \datapaths.
\smallskip

The key challenge in providing such flexibility is achieving good performance. 
Congestion control schemes traditionally process every incoming ACK, but doing so, \ccp would incur unacceptable overheads. 
We propose to tackle this problem using a new flexible batching method to summarize and communicate information between the \datapath and \ccp{} once or twice per RTT rather than on every ACK\@. 
We show that this approach can achieve behavior close to native \datapath
implementations since the fundamental time-scale for end-to-end congestion control is an RTT\@. 
To our knowledge, \ccp{} is the first such off-datapath congestion-control architecture that does not process every single packet or ACK\@. 

\section{Procedure}%
\label{sec:procedure}

We outline 360 hours of work below.

\subsection{Identify congestion control primitives} 
We will identify a set of sufficiently general primitives datapaths should expose by studying the designs and respective implementations of existing congestion control algorithms. 110 hours.

\subsection{Design a sufficiently expressive API for congestion control algorithms}
Using the set of primitives as a base, we will design an API expressive enough to implement various congestion control algorithms, and enable new ones. The design of this API is undetermined; a key question is how algorithm implementations will query the datapath for congestion control information. 110 hours.

\subsection{Evaluate the extent to which occasional modulation of congestion control decisions causes modified behavior}
We will design and implement a research prototype with which we will write implementations of existing congestion control algorithms. We will use this prototype to determine whether occasional control is sufficient for congestion control. 110 hours.

\subsection{Report Preparation}
30 hours.

\section{Facilities}
\label{sec:facilities}

We will require general-purpose computing resources as well as high-speed (\eg $10$ Gbit/s) networking equipment.


\bibliographystyle{abbrv}
\begin{small}
\bibliography{thesis}
\end{small}

\end{document}
