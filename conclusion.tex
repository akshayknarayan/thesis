The systems, techniques, and abstractions this thesis has presented are generally not \emph{necessary} for any individual application or network environment. It is entirely possible to engineer functional applications using traditional network stacks and ad-hoc communication libraries, just as it is entirely possible to re-implement congestion control algorithms for each datapath of interest. 
However, just as layered abstractions have helped us build these traditional tools to start with, this thesis has argued that new abstractions, techniques, and systems can help us scale our networked applications to meet ever-evolving demands of functionality, performance, and stability. 

The fundamental reason new abstractions are necessary is the increasing amount of heterogeneity in our networks. Rather than provide one-size-fits-all networking abstractions, modern networks continue to adopt specialization as a means to provide greater efficiency and functionality. 
While traditional abstractions such as IP continue to be valuable, new abstractions such as the Chunnel as well as CCP's measurement and control primitives will allow applications and datapaths to embrace this increasing amount of heterogeneity.
Bertha helps applications express the network features they want, which enables applications to decouple the specification of those features from their implementation. As a result, Bertha applications can defer implementation decisions to runtime, when information about the network runtime environment becomes available. Further, Bertha can provide stability and eliminate a class of bugs by checking implementations' compatibility during connection establishment.
Meanwhile, CCP decouples congestion control algorithm implementations from the complexity of datapath runtime environments, and allows them to be re-used across datapaths. 

Despite the progress Bertha and CCP offer, there remain opportunities for future work. One such direction is in understanding how datapath structures can help or hinder the performance of individual applications through the internal decisions they make. 
Given that Bertha can extend the traditional notion of the network stack upwards into what was earlier considered a part of the application's logic, how should we structure these stack components to best support the application's performance?
Within the datapaths themselves, how can we support modular structures that acknowledge the new reality of datapath heterogeneity, and allow for component reuse while preserving performance?

It will always remain possible to engineer our way around heterogeneity with bespoke implementations and one-size-fits-all performance. Indeed, this method of problem solving is agile; developers can quickly build structures that adapt to contemporary trends in hardware or user demands. However, forever building ad-hoc structures eventually results in a loss of both application as well as developer efficiency. 
It will thus remain important to follow up with abstractions and structure that can provide all three of functionality, performance, and stability. This thesis is one step along this path.
