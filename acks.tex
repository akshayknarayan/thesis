It is almost unfair that this thesis must list a single name -- \ie mine -- as its author. The ideas I describe here are the result of both formal and informal collaboration among friends and colleagues.
Any line of work will have its ups and downs; it is one's collaborators, labmates, advisors, mentors, friends, and family who provide the external context, encouragement, advice, technical support, and cheerleading that pursuing research requires. 
If anyone has come to read this looking for advice, mine is simply this: with a supportive community at your back, you can't really fail.

With this in mind, I first thank my advisors, Hari Balakrishnan and Mohammad Alizadeh, for guiding me throughout my time at MIT. Both Hari and Mohammad have a special talent for finding the core of a research project and highlighting it for everyone to see. 
One thing I have tried to learn from Hari is prioritization: the art of working on the core of a problem rather than its window dressing.
His insistence that ressearch communication should be about teaching the concepts of the work has greatly improved my communication skills.
Mohammad, meanwhile, has taught me to explain not just the easy parts of a concept, but the messy edge cases too.

I have had, and continue to have, numerous mentors other than Hari and Mohammad. First among these are Scott Shenker and Sylvia Ratnasamy, who originally encouraged me to pursue research as an undergraduate student. Scott and Sylvia, along with other members of the NetSys lab including Gautam, Justine Sherry, Aurojit Panda, Peter Gao, Rachit Agarwal, Radhika Mittal\footnote{Radhika later came to MIT as a postdoc, where we worked on Bundler together.}, and Sangjin Han showed me the joy of research and the value of collaboration. 
Additionally, after I came to MIT, Srinivas Narayana was a valuable source of research and life advice in my first two years.
I have been fortunate to work on research projects with multiple of these mentors during both my time at Berkeley and at MIT; Panda and Scott, of course, are even on this thesis's committee.
Among these mentors, Panda deserves a special mention; he is equally comfortable discussing low-level DPDK configuration parameters as he is where to take the project next, how the operating system's bootloader works, or where the best bakeries and coffeeshops in Boston are. 
Panda has been incredibly generous with his time during our time working together on Bertha, despite having numerous other projects and students of his own.
Finally, last and far from least is Arvind Krishnamurthy, who despite also having students of his own and a full schedule while on industrial leave has made the time over the past year to work with me on Bertha. I have especially appreciated Arvind's ability to identify new ways of applying a project's ideas.

Of course, I have learned not just from my mentors, but from my peers as well. In my first week at MIT I began working with Frank Cangialosi and Prateesh Goyal, and we continued this collaboration across multiple projects: Nimbus, CCP, and Bundler. I had a wonderful time working with Hongzi Mao and Parimarjan Negi\footnote{(as well as the quasi-village of authors Hongzi marshalled for the Park project)} on the Park project. 
Outside the work this thesis describes, I enjoyed my collaborations with Saksham Agarwal, Lloyd Brown, and Margarida Ferreira, from whom I learned about diverse topics from linear programming to program synthesis.

Outside of my own research, I have benefited enormously from the advice and feedback of the NMS and PDOS research groups in CSAIL at MIT. While Hari, Mohammad, and more recently Manya Ghobadi's NMS research group has been my official home, Frans Kaashoek, Nickolai Zeldovich, and Robert Morris\footnote{(and more recently, Adam Belay and Henry Corrigan-Gibbs)} welcomed me into PDOS as well. Some of the many supportive and friendly members of NMS and PDOS (who I have not already mentioned) have
vastly enriched my time at MIT include:
Sheila, Anirudh, Ravi, Frank W., Malte, Amy, Jon\footnote{My desk was next to Jon Gjengset's in my first year at MIT, and I attribute this as the reason both CCP and Bertha are written in Rust.}, Tej, Vikram, Mehrdad, Songtao, Jonathan, Derek\footnote{Jonathan and Derek warrant additional recognition as my roommates through most of my time at MIT.}, Ahmed, Anish, Vibhaa, Venkat, Arjun, Inho, Lily, Josh, Arash, Pouya, Lei, Seo Jin, Ralf, Ariel, Upamanyu, Alex, Pantea, Will, and many others.

Outside of research, my family -- Amma, Appa, Rohan, and Madhuri -- have always been uniquely able to contextualize my work and remind me of what is truly important, and my friends Sagar Karandikar, Sheevangi Pathak, Shoumik Palkar, and Paroma Varma have entertained me with topics from Formula One to whether checking bags on an airplane is a good idea\footnote{It's not.}.

My partner Deepti, who I met at MIT while we worked on CCP together\footnote{Another thing I thank Hari for: introducing me and Deepti!}, has supported me on everything from baking me cookies before a paper deadline to helping me debug my DPDK code. While some have commented that it might be hard to spend so much time with someone who not only shares a career but also is in a closely related research area, I have found that this has only helped us empathize with each other more.
I especially admire Deepti's ability to rapidly fill our social schedule with meetups as well as her technical ability to dive deep into a performance optimization.

Finally, thanks to you, for reading this thesis!
